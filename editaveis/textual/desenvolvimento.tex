\chapter[Desenvolvimento ]{Desenvolvimento}

A fim de responder as questões de pesquisa estabelecidas pelo orientador,
utilizaremos como base o \cite{swebok} . Cada resposta para uma questão
será divida em sessões, das mesma maneira, subquestões serão respondidas em
subsessões.

\section[Questão de Pesquisa 1]{Questão de Pesquisa 1}
\textbf{O que é o SWEBOK? Por quem e para que ele foi escrito?}


Como já dito na introdução, o \cite[swebok]{swebok}, de acordo com a página 29,
é um projeto mantido pela
sociedade de computação da \cite[ieee]{ieee} , utilizado como um guia para a
engenharia de software, que abrange várias áreas e competências deste campo do
conhecimento.

\section[Questão de Pesquisa 2]{Questão de Pesquisa 2}
\textbf{Baseado no SWEBOK v. 3.0, descreva o que é Projeto de Interface do
Usuário (User Interface Design).}

\subsection[Área do conhecimento (KA) ele está contido]{Área do conhecimento (KA) ele está contido}

A área de Projeto de Interface do Usuário está contida no Área de conhecimento
de projeto de software como um todo, titulada pelo \cite[swebok]{swebok}, na página 50,
de Software Desing.

\subsection[Os princípios gerais]{Os princípios gerais}
São sete princípios gerais que são instituídos por \cite{swebok}, na página 55,
na área de interface
de usuário. Eles prezam por garantir o bom uso do usuário com o sistema. Estes
princípios pregam os seguintes pontos:

O sistema não deve surpreender os usuários, com um conportamento que que não seja
esperado. Bem como deve ser de fácil aprendizado, fazendo com que seja fácil para
o usuário conseguir entender como trabalhar. Ele deve ser familiar para o nicho
de pessoas que irão utilizar o código. As operações devem ser todas consistentes,
fazendo exatamente o que se propõem a fazer para o usuário. Devem alertar e mostrar
aos usuários quais são os erros que acontecem durante as operações, possibilitando-o
refazer a operação caso ela esteja errada. Além de ser capaz de ser interativa
para vários tipos de usuários que irão utilizar o sistema.

\subsection[Os principais problemas]{Os principais problemas}
De acordo com \cite{swebok}, na página 55, Existem dois problemas que o projeto da interface de usuário deve resolver. O
primeiro deles é como o usuário deve interagir com o software? E por segundo, mas
não menos importante, como as informações devem ser apresentadas para o usuário?

\subsection[Modalidade]{Modalidades}
Para o \cite{swebok}, na página 55, as modalidades podem ser divididas e classificadas em seis estilos primários.

\begin{itemize}
    \item Questão-resposta: onde o usuário simplesmente faz uma pergunta para o
        software, recebendo uma resposta posteriormente para a respectiva pergunta.

    \item Manipulação direta: modalidade onde o usuário pode manipular objetos
        do software que estão sendo apresentados no dispositivo de saída, por
        exemplo, a tela de apresentação.

    \item Seleção por menu: este estilo garante que o usuário pode selecionar um
        comando através de um menu de vários comandos.

    \item Preenchimento de formulário: é o ato do usuário poder manipular e inserir
        uma sequência de informações atravéz de um formulário.

    \item Linguagem de comando: interpretações de comandos dados pelo usuário,
        para exercer algum tipo de processamento.

    \item Linguagem natural: um sistema de front-end capaz de interpretar um
        comando em liguagem natural do usuário para um comando do software.
\end{itemize}

\subsection[Projeto de apresentação de informações]{Projeto de apresentação de informações}
De acordo com \cite{swebok}, na página 55, seja de forma textual ou gráfica, as informações podem
ser apresentadas ao usuários. E a maneira correta é que haja divisões para quem
está interagindo com o sistema quanto à vários aspéctos. Dentre esses aspectos,
podemos destacar: limite de cores usadas, usar cores diferentes para diferenciar
assuntos, suporte de cores para diferencias tarefas, entre outros pontos.

\subsection{Processo do projeto de Interface}
\label{sub:Processo do projeto de Interface}

O processo de desenvolvimento de uma interface entre o humano e o software tem
três etapas bem defenidas, de acordo com \cite{swebok}, na página 56,
que podem se repetir
ciclicamente até que o usuário esteja satisfeito com o produto. Existem
basicamente três etapas:

\begin{itemize}
    \item Análise do usuário: nesta etapa, os projetistas analisam o usuário do
        sistema, a fim de identificar quais as necessidades daquele software.

    \item Prototipação do software: responsável por fazer um protótipo da
        interface de usuário com base na avaliação feita pelos projetistas.

    \item Evolução da interface: com base no feedback do usuário e da sua
        experiência com o protótipo, os projetistas fazem uma análise e verificam
        a necessidade de mudanças, voltando para o primeiro ponto até o cliente
        estar satisfeito.
\end{itemize}

\subsection{Localização e Internacionalização}
\label{sub:Localização e Internacionalização}

Se faz necessário que o software seja adaptável para o ambiente em que o usuário
está inserido, bem como a linguagem nativa deste.

Assim, os projetistas tem que implementar a internacionalização do software
para diferentes áreas.

\subsection{Modelos conceituais e metáforas}
\label{sub:Modelos conceituais e metáforas}

Segundo o \cite{swebok}, na página 56, métaforas com o cotidiano e com o mundo real auxilian a
deixar o software mais inteligível pelo usuário, pois este irá fazer associações
e o entendimento será mais simples.

\section{Teste de Software}
\label{sec:Teste de Software}

Para o \cite{swebok}, na página 82, testes de software é um processo dinâmico que verifica o sistema
baseado em comportamentos esperados. Para tal, o objetivo do teste é obter a
informação se o software está se comportando conforme o esperado, sendo que a
única forma de averiguar certo nível de acurácia do software é o testando.

Dessa maneira, o objetivo dos testes de usabilidade e da interação entre humano
computador é provar o quão fácil o sistema é para que o usuário aprenda a manipulá-lo.
Garantindo aprendizado com os reports de erro e mensagens de aceitação do sistema.
