\chapter[Desenvolvimento ]{Desenvolvimento}

A fim de responder as questões de pesquisa estabelecidas pelo orientador,
utilizaremos como base o REFERENCIA\_SWEBoK. Cada resposta para uma questão
será divida em sessões, das mesma maneira, subquestões serão respondidas em
subsessões.

\section[Questão de Pesquisa 1]{Questão de Pesquisa 1}
\textbf{O que é o SWEBOK? Por quem e para que ele foi escrito?}


Como já dito na introdução, o SWEBoK(Referencia) é um projeto mantido pela
sociedade de computação da IEEE(Referencia), utilizado como um guia para a
engenharia de software, que abrange várias áreas e competências deste campo do
conhecimento.

\section[Questão de Pesquisa 2]{Questão de Pesquisa 2}
\textbf{Baseado no SWEBOK v. 3.0, descreva o que é Projeto de Interface do
Usuário (User Interface Design).}

\subsection[Área do conhecimento (KA) ele está contido]{Área do conhecimento (KA) ele está contido}

A área de Projeto de Interface do Usuário está contida no Área de conhecimento
de projeto de software como um todo, titulada pelo SWEBoK(REFERENCIA) de Software
Desing.

\subsection[Os princípios gerais]{Os princípios gerais}
São sete princípios gerais que são instituídos pelo SWEBoK na área de interface
de usuário. Eles prezam por garantir o bom uso do usuário com o sistema. Estes
princípios pregam os seguintes pontos:

O sistema não deve surpreender os usuários, com um conportamento que que não seja
esperado. Bem como deve ser de fácil aprendizado, fazendo com que seja fácil para
o usuário conseguir entender como trabalhar. Ele deve ser familiar para o nicho
de pessoas que irão utilizar o código. As operações devem ser todas consistentes,
fazendo exatamente o que se propõem a fazer para o usuário. Devem alertar e mostrar
aos usuários quais são os erros que acontecem durante as operações, possibilitando-o
refazer a operação caso ela esteja errada. Além de ser capaz de ser interativa
para vários tipos de usuários que irão utilizar o sistema.

\subsection[Os principais problemas]{Os principais problemas}
Existem dois problemas que o projeto da interface de usuário deve resolver. O
primeiro deles é como o usuário deve interagir com o software? E por segundo, mas
não menos importante, como as informações devem ser apresentadas para o usuário?

\subsection[Modalidade]{Modalidades}
As modalidades podem ser divididas e classificadas em seis estilos primários.

\begin{itemize}
    \item Questão-resposta: onde o usuário simplesmente faz uma pergunta para o
        software, recebendo uma resposta posteriormente para a respectiva pergunta.

    \item Manipulação direta: modalidade onde o usuário pode manipular objetos
        do software que estão sendo apresentados no dispositivo de saída, por
        exemplo, a tela de apresentação.

    \item Seleção por menu: este estilo garante que o usuário pode selecionar um
        comando através de um menu de vários comandos.

    \item Preenchimento de formulário: é o ato do usuário poder manipular e inserir
        uma sequência de informações atravéz de um formulário.

    \item Linguagem de comando: interpretações de comandos dados pelo usuário,
        para exercer algum tipo de processamento.

    \item Linguagem natural: um sistema de front-end capaz de interpretar um
        comando em liguagem natural do usuário para um comando do software.
\end{itemize}
