\chapter[Introdução]{Introdução}

No contexto de Interação Humano Computador, faz-se importante que o usuário tenha
facilidade no uso do software desenvolvido. Então há a preocupação com garantia
que o software terá um bom nível de acessibilidade e usabilidade, a fim de que o
usuário consiga utilizar bem o produto desenvolvido. Dessa forma, a sociedade de
computação da IEEE  possui um projeto chamado SWEBoK, que é um guia de um compilado
de um corpo de conhecimento sobre a engenharia de software. Neste guia, há vários
pontos sobre engenharia de software, dentre eles, o Projeto de Interface de Usuário
que é o ponto que nos interessa.

\section[Questões de Pesquisa]{Questões de Pesquisa}
Ao longo deste trabalho, iremos discutir sobre alguns âmbitos que circundam a área
de IHC. Para guiar o nosso estudo, algumas questões foram propostas pelo orientador
André Barros. Tais questões serão dispostas a seguir:

\begin{enumerate}
    \item O que é o SWEBOK? Por quem e para que ele foi escrito?
    \item Baseado no SWEBOK v. 3.0, descreva o que é Projeto de Interface do
        Usuário (User Interface Design). Essa descrição deve conter:
        \begin{itemize}
            \item Qual área do conhecimento (KA) ele está contido;

            \item Os princípios gerais;

            \item Os problemas principais;

            \item As Modalidades;

            \item A apresentação da informação;

            \item Processo;

            \item Localização e Internacionalização;

            \item Modelos conceituais e metáforas.)
        \end{itemize}
    \item Baseado no SWEBOK v. 3.0, descreva o que é Teste de Software.
        Essa descrição deve conter a definição e o objetivo dos Testes de
        Usabilidade e Interação Humano Computador.
\end{enumerate}
